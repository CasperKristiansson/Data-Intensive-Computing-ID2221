\documentclass[12pt]{article}
\usepackage[utf8]{inputenc}
\usepackage{geometry}
\usepackage{graphicx}
\usepackage{amsmath}
\usepackage{amssymb}
\usepackage{hyperref}
\usepackage{cleveref}
\usepackage[backend=biber, sorting=none, style=ieee]{biblatex}
\addbibresource{main.bib}

\geometry{a4paper, margin=1in}

\title{Data Intensive Computing Final Summary (ID2221)}
\author{Casper Kristiansson}
\date{\today}

\begin{document}

\maketitle

\section{Summary Slides}
\href{https://id2221kth.github.io/}{https://id2221kth.github.io/}

\subsection{Introduction}
\href{https://id2221kth.github.io/slides/2023/01\_introduction.pdf}{https://id2221kth.github.io/slides/2023/01\_introduction.pdf}

\subsubsection{NIST (National Institute of Standards and Technology)}
NIST is defined by:
\begin{itemize}
    \item Five characteristics
    \item Three service models
    \item Four deployment models
\end{itemize}

\subsubsection{Cloud Characteristics}
\begin{itemize}
    \item \textbf{On-demand self-service:} A consumer can independently provision computing capabilities
    \item \textbf{Ubiquitous Network Access:} available for any network devices over the network
    \item \textbf{Resource Pooling:} Providers computing resources are pooled to serve customers
    \item \textbf{Rapid Elasticity:} Can rapidly scale on demand.
    \item \textbf{Measured Service:} Can easily measure control and report statistics of resource usage
\end{itemize}

\subsubsection{Cloud Service Models}
\begin{itemize}
    \item \textbf{SaaS:} It is like living in a hotel where vendors provide application access over a network like Gmail and GitHub.
    \item \textbf{IaaS:} Like building a new house. A vendor provides resources that a consumer can utilize, often via a customized virtual machine like EC2 or s3.
    \item \textbf{PaaS:} Building an empty house. The vendor provides hardware and development environments like the Google app engine.
\end{itemize}

\subsubsection{Deployment Models}
\begin{itemize}
    \item \textbf{Public Cloud:} AWS, Azure, GCP, etc. The main services that they provide are computing power, storage, databases, and big data analytics.
    \begin{itemize}
        \item \textbf{Storage:} File storage, block storage, and object storage
        \item \textbf{Big Data Analytics:} Data warehouse, streaming queuing.
    \end{itemize}
\end{itemize}

\subsubsection{Big Data}
\begin{itemize}
    \item Big data is characterized by 4 key attributes: \textbf{volume, variety, velocity,} and \textbf{value}.
    \item \textbf{Scaling:}
    \begin{itemize}
        \item Traditional platforms fail due to performance and need a new system that can store and process large-scale data.
        \item \textbf{Scale vertically} (up) by adding resources to a single node in a system. Usually more expensive than scaling out.
        \item \textbf{Scale horizontally} (out) is by adding more nodes to a system. Usually more challenging due to fault tolerance and software development.
    \end{itemize}
    \item \textbf{Resource Management:} Manage resources of a cluster
    \item \textbf{Distributed file systems:} Store and retrieve files in/from distributed disks
    \item \textbf{NoSQL databases:} BASE instead of ACID
    \item \textbf{Data Storage:} Store streaming data
    \item \textbf{Batch Data:} Process data-at-rest, data-parallel processing model
    \item \textbf{Streaming data:} process data-in-motion
    \item \textbf{Linked data (Graph data):} graph parallel processing, vertex-centric and edge-centric programming model
    \item \textbf{Structured data}
    \item \textbf{Machine Learning:} Data analysis supervised and unsupervised learning
\end{itemize}

\end{document}


\printbibliography

\end{document}
