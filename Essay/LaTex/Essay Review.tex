\documentclass[12pt]{article}
\usepackage[utf8]{inputenc}
\usepackage{geometry}
\usepackage{graphicx}
\usepackage{amsmath}
\usepackage{amssymb}
\usepackage{hyperref}
\usepackage{cleveref}
\usepackage[backend=biber, sorting=none, style=ieee]{biblatex}
\addbibresource{main.bib}

\geometry{a4paper, margin=1in}

\title{Opposition: Data-Intensive Computing Storage}
\date{\today}

\begin{document}

\maketitle

\noindent
\begin{minipage}{0.5\textwidth}
    \centering
    \section*{Opponents}
    Casper Kristiansson \\
    Nicole Wijkman
\end{minipage}%
\begin{minipage}{0.5\textwidth}
    \centering
    \section*{Respondents}
    Manushree Tyagi \\
    Francis Gniady \\
    Yuvraj Gurjeet Singh
\end{minipage}


\section{General Overview}

The authors have written a high-quality summary and analysis of the papers within their chosen field of computing storage. Their essay clearly communicates the importance of reliable and scalable storage systems in the context of our modern world where vast amounts of data are generated every single day.

Overall, this essay provides a well-written, comprehensive summary and opinion on the five papers regarding different storage systems. The authors highlight the importance of these solutions by addressing the challenges of storing and accessing vast amounts of data, while simultaneously discussing the strengths and weaknesses of the papers.

\section{Strengths}
\begin{itemize}
    \item \textbf{Broad Overview}: The authors did a good job providing a good introduction and overview of why the different papers within Data-Intensive Computing Storage are important. They provided a good motivation behind it discussing points like reliability, distribution, and scalability.
    \item \textbf{Analysis}: The essay clearly shows how each system works differently, especially when dealing with mistakes, sharing tasks, and writing data. The discussion of the solutions the different papers contributed was well-written and interesting to read.
    \item \textbf{Opinion}: The authors provided good opinions on both the strengths and weaknesses of the papers. This balanced approach showcases their thorough understanding of the subject.
\end{itemize}

\section{Improvements}
\begin{itemize}
    \item \textbf{References}: Improve the references. It becomes kind of hard when you just create a main reference to the different papers instead of making more direct references like pages etc. Especially in the beginning where the first sentence creates a global reference to the all different papers.
\end{itemize}
\begin{itemize}
    \item \textbf{Papers}: While discussing the five different papers I feel like there was not any deep dive into the different papers. We think it would be more interesting to discuss one paper at a time and talk more about that specific paper.
\end{itemize}

\printbibliography

\end{document}
