\documentclass[12pt]{article}
\usepackage[utf8]{inputenc}
\usepackage{geometry}
\usepackage{graphicx}
\usepackage{amsmath}
\usepackage{amssymb}
\usepackage{hyperref}
\usepackage{cleveref}
\usepackage[backend=biber, sorting=none, style=ieee]{biblatex}
\addbibresource{main.bib}

\geometry{a4paper, margin=1in}

\title{Opposition: Data-Intensive Computing Storage}
\date{\today}

\begin{document}

\maketitle

\noindent % Prevents indentation at the beginning of the line
\begin{minipage}{0.5\textwidth}
    \centering
    \section*{Opponents}
    Casper Kristiansson \\
    Nicole Wijkman
\end{minipage}%
\begin{minipage}{0.5\textwidth}
    \centering
    \section*{Respondents}
    Manushree Tyagi \\
    Francis Gniady \\
    Yuvraj Gurjeet Singh
\end{minipage}


\section{General Overview}

The authors have written a high-quality summary and analysis of the papers within their chosen field of computing storage. Their essay clearly communicates the importance of reliable and scalable storage systems in the context of our modern world where vast amounts of data are generated every single day.

Overall, this essay provides a well-written, comprehensive summary and opinion on the five papers regarding different storage systems. The authors highlight the importance of these solutions by addressing the challenges of storing and accessing vast amounts of data, while simultaneously discussing the strengths and weaknesses of the papers.

%The authors offer a well-structured summary and analysis of key papers that have formed the bedrock of modern data storage systems. The essay clearly communicates the importance of performant, reliable, and scalable storage systems in the context of tech giants that generate vast amounts of data.

% The authors offer a comprehensive summary and opinion on various seminal papers related to data storage solutions developed by major tech companies. They effectively highlight the importance of these solutions in addressing the challenges of storing and accessing vast amounts of data, while also underscoring the strengths and weaknesses of these papers.


\section{Strengths}
\begin{itemize}
    \item \textbf{Broad Overview}: The authors did a good job providing a good introduction and overview of why the different papers within Data-Intensive Computing Storage are important. They provided a good motivation behind it discussing points like reliability, distribution, and scalability.
    \item \textbf{Analysis}: The essay clearly shows how each system works differently, especially when dealing with mistakes, sharing tasks, and writing data. The discussion of the solutions the different papers contributed was well-written and interesting to read.
    \item \textbf{Opinion}: The authors provided good opinions on both the strengths and weaknesses of the papers. This balanced approach showcases their thorough understanding of the subject.
\end{itemize}

\section{Improvements}
\begin{itemize}
    \item \textbf{References}: Improve the references. It becomes kind of hard when you just create a main reference to the different papers instead of making more direct references like pages etc. Especially in the beginning where the first sentence creates a global reference to the all different papers.
\end{itemize}
\begin{itemize}
    \item \textbf{Papers}: While discussing the five different papers I feel like there was not any deep dive into the different papers. We think it would be more interesting to discuss one paper at a time and talk more about that specific paper.
\end{itemize}


%Consistency in Referencing: While the authors frequently mention the papers in their analysis, consistent citation would make the essay more academic and easier to cross-reference. For instance, the distinction between general commentary and specific findings from the papers can be made clearer with more consistent citation.

%Clarify Assumptions: Statements like "solutions ... that worked best for the authors" might be interpreted as subjective. It would be beneficial to clarify based on what criteria these solutions were deemed "best" - was it performance, ease of implementation, or some other metric?

%Expand on Financial Implications: While the financial motivations behind creating efficient storage systems are introduced, a more in-depth discussion or concrete examples of cost savings could add more depth to the analysis.

%Dive Deeper into Disadvantages: The essay predominantly focuses on the strengths and contributions of each paper. Including criticisms or known limitations of each system would present a more balanced view.

%Recommendations for Future Research: The essay could benefit from a concluding section that synthesizes the main findings and provides recommendations or questions for future research in the field of data storage systems.

%Company-Specific Focus: While the authors rightly point out that these systems were built for company-specific needs, it would be valuable to also discuss how these systems influenced or paved the way for other storage solutions in the wider industry.

%Complexity Evaluation: The mention of system complexity and the need for external supporting systems like Zookeeper is pertinent. However, the essay could have elaborated on the challenges this complexity poses for adaptation or integration into other existing systems.

%Evolving Landscape: While the essay touches on the age of the papers as a potential drawback, it would be beneficial to include some insights or examples of more recent developments or innovations in the data storage domain to provide context.

%Real-World Evaluations: The authors correctly identify the lack of real-world tests and evaluations as a shortcoming. But it would be constructive to suggest possible real-world scenarios or challenges that could have been considered in these evaluations.


\printbibliography

\end{document}
